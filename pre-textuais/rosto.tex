%
% ********** Página de Rosto
%

% titlepage gera páginas sem numeração
\begin{titlepage}

\begin{center}

\small

% O comando @{} no ambiente tabular x é para criar um novo delimitador
% entre colunas que não a barra vertical | que é normalmente utilizada.
% O delimitador desejado vai entre as chaves. No exemplo, não há nada,
% de modo que o delimitador é vazio. Este recurso está sendo usado para
% eliminar o espaço que geralmente existe entre as colunas
\begin{tabularx}{\linewidth}{@{}l@{}C@{}r@{}}
% A figura foi colocada dentro de um parbox para que fique verticalmente
% centralizada em relação ao resto da linha
\parbox[c]{3cm}{\includegraphics[width=\linewidth]{./figuras/UFRN}} &
\begin{center}
\textsf{\textsc{Universidade Federal do Rio Grande do Norte\\
Escola de Ciências e Tecnologia\\
Programa de Pós-Graduação em Ciência, tecnologia e inovação}}
\end{center} &
\parbox[c]{3cm}{\includegraphics[width=\linewidth]{./figuras/ppgcti-abreveatura}}
\end{tabularx}

% O vfill é um espaço vertical que assume a máxima dimensão possível
% Os vfill's desta página foram utilizados para que o texto ocupe
% toda a folha
\vfill

\LARGE

\textbf{Sobre a Preparação de Propostas de Tema, Dissertações
e Teses no Programa de Pós-Graduação em Engenharia Elétrica e de Computação da UFRN}

\vfill

\Large

\textbf{Danilo Mikael Costa Barros}

\vfill

\normalsize

Orientador: Prof. Dr. Gláucio Bezerra Brandão
% Se não houver co-orientador, comente a próxima linha
\\[2ex] Co-orientador: Prof. Dr. Beltrano Catandura do Amaral

\vfill

\hfill
\parbox{0.5\linewidth}{\textbf{%
% Descomente as opções que se aplicam ao seu caso
%Proposta de Tema para Qualificação}
Dissertação de Mestrado}
apresentada ao Programa de Pós-Graduação em Ciência, Tecnologia e Inovação da Escola de Ciências, Tecnologia e Inovação - EC\&T da Universidade Federal do Rio Grande do Norte - UFRN,
como parte dos requisitos para obtenção do título de Mestre em Ciências, Tecnologia e Inovação.}
%Doutor em Ciências.}

\vfill

\large

% Este número de ordem deve ser obtido na coordenação do PPgEE
% Corresponde ao número seqüencial da sua tese ou dissertação:
% por exemplo, a 25ª tese de doutorado terá o número de ordem D25
% Evidentemente, este dado não existe para propostas de tema, caso
% em que a próxima linha deve ser comentada.
Número de ordem PPgCTI: M000

Natal, RN, maio de 2018

\end{center}

\end{titlepage}
