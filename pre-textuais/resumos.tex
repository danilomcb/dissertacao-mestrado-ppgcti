%
% ********** Resumo
%

% Usa-se \chapter*, e não \chapter, porque este "capítulo" não deve
% ser numerado.
% Na maioria das vezes, ao invés dos comandos LaTeX \chapter e \chapter*,
% deve-se usar as nossas versões definidas no arquivo comandos.tex,
% \mychapter e \mychapterast. Isto porque os comandos LaTeX têm um erro
% que faz com que eles sempre coloquem o número da página no rodapé na
% primeira página do capítulo, mesmo que o estilo que estejamos usando
% para numeração seja outro.
\mychapterast{Resumo}

O risco de um investimento em um parque eólico esta fortemente atrelado as diferenças entre a produção efetiva e a produção estimada durante a elaboração do projeto. O acompanhamento do funcionamento dos parques eólicos se faz necessário para que seja garantido o bom desempenho dos mesmo. Neste trabalho será discutido a criação de um produto para uma setor produtivo carente de tecnologia nacional. Será apresentado o Windbox, ferramenta que possibilita o monitoramento contínuo de parques eólicos e que visa o aumento da eficiência desses parques. Além disso será enfatizado a trajetória de criação desta solução assim como os resultados obtidos até o presente momento e as perspectivas futuras.

\vspace{1.5ex}

{\bf Palavras-chave}: Inovação, Processamento de Dados, Energia Eólica,
Coleta de Dados Industriais.

%
% ********** Abstract
%
\mychapterast{Abstract}

The investment risk of a wind farm is heavily bound to the differences between the effective production and the estimated production during the project elaboration.
Monitoring the functioning of wind farms is necessary in order to ensure their good performance. This paper discusses the creation of a product to a productive sector that necesses national technology. It will be presented the Windbox, a tool that allows the continuous monitoring of wind farms and that seeks efficience gain of such wind farms. In addition it will be emphasized the trajectory taken to create this solution as well as the results obtained up to the present moment and future perspectives.

\vspace{1.5ex}

{\bf Keywords}: Innovation, Data Processing, Wind Power,
Industrial Data Collection.
