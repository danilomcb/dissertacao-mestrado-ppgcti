%%
%% Capítulo 1: Modelo de Capítulo
%%

% Está sendo usando o comando \mychapter, que foi definido no arquivo
% comandos.tex. Este comando \mychapter é essencialmente o mesmo que o
% comando \chapter, com a diferença que acrescenta um \thispagestyle{empty}
% após o \chapter. Isto é necessário para corrigir um erro de LaTeX, que
% coloca um número de página no rodapé de todas as páginas iniciais dos
% capítulos, mesmo quando o estilo de numeração escolhido é outro.
\mychapter{Introdução}
\label{Cap:Introducao}

O aproveitamento energético de recursos eólicos no Brasil atravessa um momento de expansão em participação na matriz de energia elétrica nacional, chegando ao final de 2017 a um percentual de participação de 8,1\% segundo a \citeasnoun{boletim-anual-geracao-2017}.

Em outubro de 2017, o Brasil chegou a 12,33 GW de capacidade instalada de energia eólica, com 491 parques eólicos. Ultrapassamos a Itália na capacidade instalada de energia e alcaçamos o 9º lugar no ranking mundial~\cite{abeeolica-boletim-mensal-geracao-out-2017}. A estimativa do governo, que consta no Plano Decenal de Expansão de Energia, é que em 2023 a produção eólica brasileira deverá responder por 11,7\% de toda a produção gerada pelo país. É o tipo de produção de energia que mais cresce no país.

No total acumulado dos anos, o Brasil já tem mais de R\$ 70 bilhões de investimento acumulado e 160 mil empregos em toda a cadeia produtiva. Além de ser uma fonte com baixíssimo impacto de implantação, a eólica não emite CO\textsubscript{2}, o total de emissões evitadas em 2016 foi de 17,81 milhões de toneladas, o equivalente à emissão anual de cerca de 12 milhões de automóveis. Isso vai ajudar o país a cumprir com suas metas de redução da emissão de CO\textsubscript{2} e diversificação da matriz com fontes renováveis complementares.

O Nordeste é referência em produção eólica no Brasil. No dia 23 de junho de 2018 a energia produzida pelos ventos foi responsável por atender 71\% da carga do subsistema do Nordeste, batendo o novo recorde de geração média diária com 7.137 MW médios. Devido a esse bom desempenho, o Nordeste tem sido exportador de energia para o Sudeste/Centro-Oeste \cite{ons-recorde-geracao}.

No ano de 2016 o Rio Grande do Norte foi o estado que apresentou maior geração com 10,59 TWh de energia produzida, representando 33,02\% da geração nacional. Com 125 parques eólicos e 3,4 GW de capacidade instalada de energia eólica, o Rio Grande do Norte é atualmente o primeiro do ranking. O Estado tem cerca de 1.700 aerogeradores. Se considerarmos o que já está contratado para o RN, serão mais 1,2 GW e mais 50 parques até 2020 ~\cite{estados-capacidade-instalada}.

A energia eólico é uma fonte renovável de energia, limpa e inesgotável que sem dúvida, vem crescendo e assumindo uma importante posição no Brasil. Entretanto, para o aproveitamento deste recurso energético são necessários altos níveis de investimentos em parques eólicos. Estes investimento, por sua vez, são rentabilizados de acordo com o desempenho dos aerogeradores que compõem os parques. Procurando minimizar o risco financeiro, os investidores se baseiam em estimativas energéticas do período de produção das usinas, para assim, poder ter uma noção no presente, do seu faturamento futuro. Contudo, ao se comparar a geração prevista com a geração real de um parque, percebe-se que, normalmente, a produção é inferior ao esperado ~\cite{ons2015}. Esses desvios são ocasionados por fatores diversos, em sua maioria relacionados à falhas nos aerogeradores e no sistema elétrico dos parques. Desta forma, é necessário compreender as razões pelas quais existem diferenças entre a geração real obtida e a estimada.

Para impedir os problemas relacionados à falhas, é necessário um planejamento criterioso das manutenções e este tipo de planejamento só é possível mediante o acompanhamento efetivo do desempenho dos parques. Foi pensando nisso que surgiu o Windbox, uma ferramenta na nuvem, no qual os gestores de parques eólicos podem analisar e monitorar o desempenho e a disponibilidade dos seus aerogeradores, assim como, identificar falhas recorrentes e planejar com mais precisão as manutenções. O principal propósito do Windbox é possibilitar o contínuo monitoramento do funcionamento de parques eólicos permitindo detectar desempenhos insatisfatórios, identificar e corrigir as razões destes desvios e, assim, evitar perdas de receita. Além disso, o Windbox possibilita que seja formada uma base de dados e de ensinamentos que podem ser úteis no projeto, construção e operação de outros parques.

Neste trabalho será discutido o desenvolvimento de um projeto de inovação a partir da colaboração de uma empresa privada com uma ICT (Instituição Ciêntifica e tecnológica). Serão demonstrados como o Windbox, um projeto desenvolvido pela LogAp Sistemas em conjunto com o CTGAS-ER (Centro de Tecnologia do Gás e Energias Renováveis), foi criado e introduzido no mercado.

O presente texto está dividido em 6 capítulos. No Capítulo 2 serão apresentados os conceitos básicos que auxiliam o desenvolvimento do trabalho, além de uma visão geral do cenário e da problemática, de maneira a esclarecer as principais características do setor eólico brasileiro e mostrar como o Windbox agrega valor a este mercado.

O capítulo 3 apresentará o desenvolvimento da solução, mostrado todas as suas fases, desde a sua idealização e prototipação até o seu estado atual, apresentando também as dificuldades encontradas e superadas ao longo do caminho.

No Capítulo 4 será dedicado a gestão da inovação, no qual será discutida a atuação do CInO (\textit{Chief Innovation Officer}) e a sua importância para o desenvolvimento do projeto apresentado.

Por fim, o Capítulo 5, serão apresentados os resultados do projeto, considerações finais e a perspectivas para o futuras.


%\section{Organização do texto}
%
%O fecho do capítulo introdutório muitas vezes apresenta uma ideia
%global do trabalho, mostrando o que vai ser tratado nos capítulos
%subsequentes.



