
\mychapter{\textit{CHIEF INNOVATION OFFICER} (CINO)}
\label{Cap:trajetoriaFormacaoCINO}

O empreendedorismo surgiu como uma oportunidade em meados de 2012, um ano após a graduação em Engenharia de Computação e Automação, quando, junto a colegas de trabalho da RN Tecnologia, foi criada a ideia de desenvolver um historiador de dados para processos industriais. A ideia foi amadurecendo no decorrer do tempo e outros colegas entraram para o projeto. O projeto se iniciou com quatro engenheiros de computação em parceria com uma empresa, a Logique Sistema. No início os quatro desenvolviam o projeto em paralelo com os seus trabalhos. 

Em 2013 foi lançado um edital de incubação do Inova Metrópole, o qual decidimos participar. Felizmente a LogAp Sistemas foi aprovada no edital e deu o primeiro passo da sua jornada, já com a equipe societária com dedicação exclusiva na empresa. No mesmo ano, conseguimos mais força para o projeto. Fomos beneficiados com uma verba de fomento proveniente do progama RHAE do CnPq. O programa é destinado à inserção de mestres e doutores em empresas privadas e nos proporcionou inserir mais talentos no projeto. Outro impulso que recebemos foi com o Edital do Sebraetec em 2014 no qual o Sebrae investiu um valor de R\$ 40.000,00 em contra partida de R\$ 10.000,00 da empresa. Esse edital possibilitou que a LogAp realizasse diversas consultorias em gestão e vendas, um dos nossos principais problemas da época.

Em 2016, com dificuldades de penetrar no pequeno mercado industrial potiguar e observando o crescente crescimento do mercado de geração de energia eólica (tanto no Brasil, como no Rio Grande do Norte), a LogAp decidiu realizar uma análise de mercado afim de obter dados sobre este segmento, verificar em qual contexto poderia atuar e saber qual o potencial do seu público-alvo nesse setor. Como na época nenhum dos membros da equipe da Logap possuía \textit{know-how} nesse setor, foi necessário buscar essa expertise. Nesse momento a LogAp buscou auxilio do CTGAS-ER (Centro de Tecnologias do Gás e Energias Renováveis), centro de referência em energia eólico no Brasil. 

Em 2017 uma das principais metas do projeto era a viabilização financeira para que fosse possível aumentar a equipe do projeto e acelerar o processo de desenvolvimento e comercialização da solução. Nesse ano, o BNB (Banco do Nordeste) lançou um edital de fomento, o FUNDECI - Fundo de Desenvolvimento Econômico, Científico, Tecnológico e de Inovação. No mesmo ano foi lançado o edital CNPq MCTIC/SETEC, que visa apoiar projetos de Pesquisa e Desenvolvimento (P\&D). A LogAp foi a única empresa do Rio Grande do Norte que foi contemplada nestes dois editais de fomento e conseguiu com isso um investimento de quase 500 mil reais, para 2 anos de desenvolvimento do Windbox. 

Na tabela \ref{Tab:editaisFomento}, são listados os editais de fomento que a LogAp Sistemas ganhou nos últimos anos.

\begin{table}[htbp]
\begin{tabularx}{\linewidth}{|X|X|X|X|} \hline
\textbf{Edital de fomento} & \textbf{Orgão} & \textbf{Projeto} & \textbf{Ano} \\ \hline
RHAE & CNPq & Athenas Historian & 2013\\ \hline
SEBRAETEC & Sebrae & Athenas Historian & 2014 \\ \hline
FUNDECI & BNB & Windbox & 2017 \\ \hline
MCTIC/SETEC & CNPq & Windbox & 2017 \\ \hline
\end{tabularx}
\caption{Editais de fomento que a LogAp Sistema ganhou.}
\label{Tab:editaisFomento}
\end{table}

% Falar em que o curso ajudou.
% Falar da especialização técnica em eólica.

Um dos pontos fracos da empresa, desde a sua formação inicial (quatro engenheiros), era a habilidade de gerir negócios inovadores. A entrada no programa de mestrado foi essencial para mudar a forma de gerir a empresa e de se posicionar no mercado. Através das ferramentas apresentadas no andamento do curso, o modelo de negócio do Windbox foi criando e aperfeiçoado. Um exemplo é a criação do canvas de proposta de valor, no qual foi possível ter uma visão mais profunda sobre o nosso seguimento de cliente e do valor que estava sendo entregue a ele. 

Após ter pago todas as matérias do mestrado, foi tomada a decisão de reavaliar todas as metodologias e ferramentas aprendidas e aplica-las na empresa. Algumas das que mais causaram impacto e são utilizadas até hoje são: criação do \textit{Business Model Canvas}, criação de \textit{Personas} e ICPs, criação de canvas de proposta de valor, técnicas para validação de produto e de gestão de projetos. 

Um passo importante que foi tomando graças a discussões geradas no ambiente das aulas, foi o de ter um membro especialista em energia eólica dentro da equipe executora do Windbox. Até então, todo o \textit{know-how} técnico em energia eólica era da equipe do CTGAS-ER e isso, na maioria das vezes, gerava um atraso no desenvolvimento da solução, devido a problema de alinhamento de agendas para reuniões entre os dois grupo. Como forma de mitigar esse problema, surgiu a ideia de um dos sócios da empresa fazer o curso de Especialização em Energia Eólica do Senai, para que a maior parte das dúvidas da equipe de desenvolvimento fossem tiradas sem a necessidade de reuniões com a equipe do CTGAS-ER. O sócio em questão foi o autor que aqui escreve, que hoje, após o término da especialização, melhorou a sua visão sobre o mercado de eólica e adquiriu o conhecimento necessário para a execução mais assertiva do projeto.

