%%
%% Capítulo 5: Conclusões
%%

\mychapter{Considerações finais}
\label{Cap:consideracoesFinais}

Ao longo desse trabalho foi possível evidenciar o amadurecimento de uma empresa tanto do ponto de vista técnico e científico quanto do ponto de vista mercadológico e o quanto uma parceria estratégica é importante para o desenvolvimento de um projeto. Apresentou-se o projeto Windbox, bem como um pouco do sua história e do mercado que ele esta inserido. 

O aumento do consumo de energia no mundo e a busca cada vez mais forte sobre fontes de energia limpa e renovável fez com que o mercado de energia eólica crescesse rapidamente no Brasil. O problema é que apesar do crescimento acelerado, a maioria dos parques eólicos ainda são carentes de tecnologia. Soluções como o Windbox ainda são novidades no nosso país o que acaba sendo um ponto forte para a sua comercialização.

A implantação de parques eólicos ainda exige um alto nível de investimento. Com o crescimento do mercado e da concorrência, o valor de remuneração obtido nesses empreendimentos tem sido cada vez menor, fazendo com que os investidores não possam tolerar qualquer problemas no desempenho dos aerogeradores e buscar ao máximo reduzir os riscos associados ao negócio. O Windbox surge nesse cenário como uma ferramenta que auxilia os gestores dos parque na minimização desses riscos associados ao projeto. Isso através do monitoramento contínuo do funcionamento dos parques eólicos em busca de desempenhos insatisfatórios e das causas desses problemas.

O projeto já foi validado em mais de 11 parque eólicos e hoje encontra-se com um piloto ativo e monitorando 92 aerogeradores de um complexo eólico. Através dos \textit{feedback} desses pilotos a ferramenta vem crescendo e se tornando cada vez mais indispensável para os parques eólicos brasileiros. 

A curto e médio prazo objetiva-se fortalecer mais o setor comercial da empresa para que a solução possa ganhar abrangência nacional e ajudar os gestores de parques eólicos a melhorarem cada vez as suas produções energéticas.
